\documentclass{article}
\usepackage[utf8]{inputenc}
\usepackage[a4paper, total={7.5in, 11in}]{geometry}
\setlength{\parindent}{0pt}
\begin{document}
\huge

\section*{Manipulation des fractions}
\subsection*{Exercice 1}
Simplifier pour obtenir une fraction irréductible.\\
exemple) $\frac{15}{25}=\frac{15\div5}{25\div5}=\frac{3}{5}$\\ 
\vspace{10 mm}
a) $\frac{147}{42}=$\\ 
\vspace{10 mm}
b) $\frac{3}{6}=$\\ 
\vspace{10 mm}
c) $\frac{35}{70}=$\\ 
\vspace{10 mm}
d) $\frac{18}{28}=$\\ 
\vspace{10 mm}
e) $\frac{4}{42}=$\\ 
\vspace{10 mm}
f) $\frac{98}{49}=$\\ 
\vspace{10 mm}
g) $\frac{10}{30}=$\\ 
\vspace{10 mm}
h) $\frac{105}{25}=$\\ 
\vspace{10 mm}
i) $\frac{10}{105}=$\\ 
\vspace{10 mm}
j) $\frac{98}{14}=$\\ 
\vspace{10 mm}
k) $\frac{6}{18}=$\\ 
\vspace{10 mm}
l) $\frac{15}{9}=$\\ 
\vspace{10 mm}
\newpage
\subsection*{Exercice 2}
Sommer puis simplifier pour obtenir une fraction irréductible.\\
exemple) $\frac{-2}{147}+\frac{16}{147}=\frac{-2+16}{147}=\frac{14}{147}=\frac{14\div7}{147\div7}=\frac{2}{21}$\\ 
\vspace{10 mm}
a) $\frac{48}{2}+\frac{-6}{2}=$\\ 
\vspace{10 mm}
b) $\frac{17}{12}+\frac{1}{12}=$\\ 
\vspace{10 mm}
c) $\frac{-6}{21}+\frac{20}{21}=$\\ 
\vspace{10 mm}
d) $\frac{109}{105}+\frac{66}{105}=$\\ 
\vspace{10 mm}
e) $\frac{-5}{4}+\frac{15}{4}=$\\ 
\vspace{10 mm}
f) $\frac{37}{70}+\frac{13}{70}=$\\ 
\vspace{10 mm}
g) $\frac{-7}{49}+\frac{42}{49}=$\\ 
\vspace{10 mm}
h) $\frac{40}{12}+\frac{-10}{12}=$\\ 
\vspace{10 mm}
i) $\frac{75}{63}+\frac{30}{63}=$\\ 
\vspace{10 mm}
j) $\frac{26}{18}+\frac{121}{18}=$\\ 
\vspace{10 mm}
k) $\frac{-5}{4}+\frac{15}{4}=$\\ 
\vspace{10 mm}
l) $\frac{42}{75}+\frac{133}{75}=$\\ 
\vspace{10 mm}
\newpage
\subsection*{Exercice 3}
Mettre au même dénominateur pour sommer puis simplifier pour obtenir une fraction irréductible.\\
exemple) $\frac{-5}{3}+\frac{25}{6}=\frac{-5\times2}{3\times2}+\frac{25}{6}=\frac{-10}{6}+\frac{25}{6}=\frac{15}{6}=\frac{15\div3}{6\div3}=\frac{5}{2}$\\ 
\vspace{10 mm}
a) $\frac{2}{10}+\frac{14}{70}=$\\ 
\vspace{10 mm}
b) $\frac{-1}{7}+\frac{13}{42}=$\\ 
\vspace{10 mm}
c) $\frac{2}{5}+\frac{99}{15}=$\\ 
\vspace{10 mm}
d) $\frac{3}{10}+\frac{110}{50}=$\\ 
\vspace{10 mm}
e) $\frac{5}{21}+\frac{-3}{42}=$\\ 
\vspace{10 mm}
f) $\frac{-5}{1}+\frac{21}{3}=$\\ 
\vspace{10 mm}
g) $\frac{3}{10}+\frac{29}{70}=$\\ 
\vspace{10 mm}
h) $\frac{-2}{3}+\frac{12}{12}=$\\ 
\vspace{10 mm}
i) $\frac{3}{4}+\frac{-14}{28}=$\\ 
\vspace{10 mm}
j) $\frac{7}{7}+\frac{294}{49}=$\\ 
\vspace{10 mm}
k) $\frac{-6}{21}+\frac{15}{42}=$\\ 
\vspace{10 mm}
l) $\frac{12}{49}+\frac{18}{98}=$\\ 
\vspace{10 mm}
\newpage
\section*{Corrections}
\subsection*{Exercice 1 corrigé}
a) $\frac{147}{42}=\frac{147\div21}{42\div21}=\frac{7}{2}$\\ 
\vspace{10 mm}
b) $\frac{3}{6}=\frac{3\div3}{6\div3}=\frac{1}{2}$\\ 
\vspace{10 mm}
c) $\frac{35}{70}=\frac{35\div35}{70\div35}=\frac{1}{2}$\\ 
\vspace{10 mm}
d) $\frac{18}{28}=\frac{18\div2}{28\div2}=\frac{9}{14}$\\ 
\vspace{10 mm}
e) $\frac{4}{42}=\frac{4\div2}{42\div2}=\frac{2}{21}$\\ 
\vspace{10 mm}
f) $\frac{98}{49}=\frac{98\div49}{49\div49}=\frac{2}{1}=2$\\ 
\vspace{10 mm}
g) $\frac{10}{30}=\frac{10\div10}{30\div10}=\frac{1}{3}$\\ 
\vspace{10 mm}
h) $\frac{105}{25}=\frac{105\div5}{25\div5}=\frac{21}{5}$\\ 
\vspace{10 mm}
i) $\frac{10}{105}=\frac{10\div5}{105\div5}=\frac{2}{21}$\\ 
\vspace{10 mm}
j) $\frac{98}{14}=\frac{98\div14}{14\div14}=\frac{7}{1}=7$\\ 
\vspace{10 mm}
k) $\frac{6}{18}=\frac{6\div6}{18\div6}=\frac{1}{3}$\\ 
\vspace{10 mm}
l) $\frac{15}{9}=\frac{15\div3}{9\div3}=\frac{5}{3}$\\ 
\vspace{10 mm}
\newpage
\subsection*{Exercice 2 corrigé}
a) $\frac{48}{2}+\frac{-6}{2}=\frac{48+-6}{2}=\frac{42}{2}=\frac{42\div2}{2\div2}=\frac{21}{1}=21$\\ 
\vspace{10 mm}
b) $\frac{17}{12}+\frac{1}{12}=\frac{17+1}{12}=\frac{18}{12}=\frac{18\div6}{12\div6}=\frac{3}{2}$\\ 
\vspace{10 mm}
c) $\frac{-6}{21}+\frac{20}{21}=\frac{-6+20}{21}=\frac{14}{21}=\frac{14\div7}{21\div7}=\frac{2}{3}$\\ 
\vspace{10 mm}
d) $\frac{109}{105}+\frac{66}{105}=\frac{109+66}{105}=\frac{175}{105}=\frac{175\div35}{105\div35}=\frac{5}{3}$\\ 
\vspace{10 mm}
e) $\frac{-5}{4}+\frac{15}{4}=\frac{-5+15}{4}=\frac{10}{4}=\frac{10\div2}{4\div2}=\frac{5}{2}$\\ 
\vspace{10 mm}
f) $\frac{37}{70}+\frac{13}{70}=\frac{37+13}{70}=\frac{50}{70}=\frac{50\div10}{70\div10}=\frac{5}{7}$\\ 
\vspace{10 mm}
g) $\frac{-7}{49}+\frac{42}{49}=\frac{-7+42}{49}=\frac{35}{49}=\frac{35\div7}{49\div7}=\frac{5}{7}$\\ 
\vspace{10 mm}
h) $\frac{40}{12}+\frac{-10}{12}=\frac{40+-10}{12}=\frac{30}{12}=\frac{30\div6}{12\div6}=\frac{5}{2}$\\ 
\vspace{10 mm}
i) $\frac{75}{63}+\frac{30}{63}=\frac{75+30}{63}=\frac{105}{63}=\frac{105\div21}{63\div21}=\frac{5}{3}$\\ 
\vspace{10 mm}
j) $\frac{26}{18}+\frac{121}{18}=\frac{26+121}{18}=\frac{147}{18}=\frac{147\div3}{18\div3}=\frac{49}{6}$\\ 
\vspace{10 mm}
k) $\frac{-5}{4}+\frac{15}{4}=\frac{-5+15}{4}=\frac{10}{4}=\frac{10\div2}{4\div2}=\frac{5}{2}$\\ 
\vspace{10 mm}
l) $\frac{42}{75}+\frac{133}{75}=\frac{42+133}{75}=\frac{175}{75}=\frac{175\div25}{75\div25}=\frac{7}{3}$\\ 
\vspace{10 mm}
\newpage
\subsection*{Exercice 3 corrigé}
a) $\frac{2}{10}+\frac{14}{70}=\frac{2\times7}{10\times7}+\frac{14}{70}=\frac{14}{70}+\frac{14}{70}=\frac{28}{70}=\frac{28\div14}{70\div14}=\frac{2}{5}$\\ 
\vspace{10 mm}
b) $\frac{-1}{7}+\frac{13}{42}=\frac{-1\times6}{7\times6}+\frac{13}{42}=\frac{-6}{42}+\frac{13}{42}=\frac{7}{42}=\frac{7\div7}{42\div7}=\frac{1}{6}$\\ 
\vspace{10 mm}
c) $\frac{2}{5}+\frac{99}{15}=\frac{2\times3}{5\times3}+\frac{99}{15}=\frac{6}{15}+\frac{99}{15}=\frac{105}{15}=\frac{105\div15}{15\div15}=\frac{7}{1}=7$\\ 
\vspace{10 mm}
d) $\frac{3}{10}+\frac{110}{50}=\frac{3\times5}{10\times5}+\frac{110}{50}=\frac{15}{50}+\frac{110}{50}=\frac{125}{50}=\frac{125\div25}{50\div25}=\frac{5}{2}$\\ 
\vspace{10 mm}
e) $\frac{5}{21}+\frac{-3}{42}=\frac{5\times2}{21\times2}+\frac{-3}{42}=\frac{10}{42}+\frac{-3}{42}=\frac{7}{42}=\frac{7\div7}{42\div7}=\frac{1}{6}$\\ 
\vspace{10 mm}
f) $\frac{-5}{1}+\frac{21}{3}=\frac{-5\times3}{1\times3}+\frac{21}{3}=\frac{-15}{3}+\frac{21}{3}=\frac{6}{3}=\frac{6\div3}{3\div3}=\frac{2}{1}=2$\\ 
\vspace{10 mm}
g) $\frac{3}{10}+\frac{29}{70}=\frac{3\times7}{10\times7}+\frac{29}{70}=\frac{21}{70}+\frac{29}{70}=\frac{50}{70}=\frac{50\div10}{70\div10}=\frac{5}{7}$\\ 
\vspace{10 mm}
h) $\frac{-2}{3}+\frac{12}{12}=\frac{-2\times4}{3\times4}+\frac{12}{12}=\frac{-8}{12}+\frac{12}{12}=\frac{4}{12}=\frac{4\div4}{12\div4}=\frac{1}{3}$\\ 
\vspace{10 mm}
i) $\frac{3}{4}+\frac{-14}{28}=\frac{3\times7}{4\times7}+\frac{-14}{28}=\frac{21}{28}+\frac{-14}{28}=\frac{7}{28}=\frac{7\div7}{28\div7}=\frac{1}{4}$\\ 
\vspace{10 mm}
j) $\frac{7}{7}+\frac{294}{49}=\frac{7\times7}{7\times7}+\frac{294}{49}=\frac{49}{49}+\frac{294}{49}=\frac{343}{49}=\frac{343\div49}{49\div49}=\frac{7}{1}=7$\\ 
\vspace{10 mm}
k) $\frac{-6}{21}+\frac{15}{42}=\frac{-6\times2}{21\times2}+\frac{15}{42}=\frac{-12}{42}+\frac{15}{42}=\frac{3}{42}=\frac{3\div3}{42\div3}=\frac{1}{14}$\\ 
\vspace{10 mm}
l) $\frac{12}{49}+\frac{18}{98}=\frac{12\times2}{49\times2}+\frac{18}{98}=\frac{24}{98}+\frac{18}{98}=\frac{42}{98}=\frac{42\div14}{98\div14}=\frac{3}{7}$\\ 
\vspace{10 mm}
\newpage
\end{document}
